\chapter{Project closure process}
\label{chap:apendiceB}

\section{Learned lessons}
In a research project it is vital to take note of all the work carried out and the results obtained. Having tracking codes of the different deliverables simplifies the development and knowledge of the history of each one.

Keeping the project monitoring files updated day by day helps to improve the quantity and quality of the information that is released.

Regarding to the project carried out, the following points are listed to be taken into account for future work:

$\bullet$ Before starting the printer, check that the printing area is free of obstacles.

$\bullet$ Keep the cartridges upright, with the head down, to prevent air from entering the reservoir channel.

$\bullet$ If the substrate is attached to the platen with adhesive tape, the thickness of it must be taken into account to add in the printing configurations.

$\bullet$ Before printing, check that the ejectors are working correctly and the ink drops have an aligned flight.

$\bullet$ Control the condition of the cleaning pad periodically, if the head begins to show ink drops on its base, it may be due to saturation of the drying material.