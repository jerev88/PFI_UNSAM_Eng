\chapter{Conclusions}

\section{Conclusions of the carried out work}
$\bullet$ Throughout this project, concentration and effort were dedicated to the improvement of a \textbf{biosensor} printed in carbon by silk screen printing using gold nanoparticles ink printed by the \textit{Inkjet} method. In turn, it seeks to achieve scalable manufacturing so that, if it is feasible, it can industrialize manufacturing.

$\bullet$ It is concluded that the gold nanoparticles ink has an acceptable anchorage with carbon ink, achieving the designs with acceptable precision. However, on a rough substrate (\textit{Valox}) does not achieve sufficient anchorage to be able to maintain the desired design.

$\bullet$ For a correct curing of each ink layer, it must be in \textit{Hot Plate} for at least 80 minutes at 80ºC.

$\bullet$ Two layers of ink with gold nanoparticles printed by \textit{Inkjet} have a thickness 3 times less than that of a carbon layer printed by screen printing.

$\bullet$ The resistivity of two cured layers is approximately 5 times greater than that of pure gold.

$\bullet$ The electrochemical capacity of a carbon \emph{WE} with 2 layers of cured gold nanoparticles is comparable to that of a gold \emph{WE} deposited by \textit{Sputtering}. Recalling the Randles-Sevcik equation \ref{ecuacion3}, the difference can be attributed, in part, to the difference in the effective areas of the \emph{WE}. Although geometrically identical (1 mm in diameter), the roughness generates a difference in the effective value used in the calculations. The advantages of \textit{inkjet} printing are ease of manufacture and considerably lower cost.

$\bullet$ The printing of two layers of gold nanoparticles ink by \textit{Inkjet} on a smooth substrate at the macroscopic level such as PET, approaches the cyclical voltammetry curve of the gold deposited by \textit{sputtering} on silicon, a smooth surface at the nanometric level.

$\bullet$ Correct droplet ejection and definition of droplet spacing were achieved for SU-8 dielectric ink. This allowed to make the prints with the desired design and motivated to continue with the development of this ink for future projects.

\section{Future work}
The present project on biosensors printed by \textit{Inkjet} method promotes, at least, two immediate steps to develop. The first linked to the composition and printing of carbon ink, to become independent of the intermediate step of screen printing. The second, and with greater impact, achieve the composition of a dielectric ink and its printing by \textit{Inkjet} method for the formation of microcuvettes that will contain the sample to be analyzed on the electrode, without having to go through other manufacturing processes.

In the Experimental Development section (Capítulo 3, \hyperref[sec:tinta_dielec]{section 3.5}) the formulation of the dielectric ink and the first test impressions with it are mentioned, which should be taken into account for the continuity of the technology development and the next improvements in the biosensors.

\nocite{Banica}
\nocite{Prudenziati}
\nocite{Voros}
\nocite{Poc}
\nocite{PosterPoc1}
\nocite{DMPDatasheet}
\nocite{AgParticlesDimatix1}
\nocite{AgParticlesDimatix2}